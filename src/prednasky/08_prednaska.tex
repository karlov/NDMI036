\section{\texorpdfstring{Konečné projektivní prostory}{Konečné projektivní prostory}}
% 9 predn

\begin{definition}[Konečný projektivní prostor, kolinearita]
    $(P,\mathcal{L})$ takový, že $P$ je konečná množina bodů a $\mathcal{L}$ je množinový systém (přímek) na $P$, je konečný projektivní prostor, splňuje-li axiomy A1, A2, A3.

    Dále body $x\neq y\neq z\neq x$ takové, že $x,y,z\in l\in\mathcal{L}$, nazveme kolineární.

    \begin{itemize}
        \item[(A1)] $\forall x\neq y\in P\exists! l\in \mathcal{L}: x,y\in l$
        \item[(A2)] $\forall l\in\mathcal{L}: |l|\geq 3$
        \item[(A3)]  $\forall a,b,c: a\neq b\neq c\neq a, a,b,c$ nekolineární: $\forall x\in ac, y\in bc\exists z\in ab$ takový, že $x,y,z$ jsou kolineární.
    \end{itemize}
\end{definition}
\begin{definition}[Podprostor]
    Je-li $(P,\mathcal{L})$ konečná geometrie, pak $U\subseteq P$ je podprostor, jestliže $\forall x\neq y\in U: xy\subseteq U$.
\end{definition}
\begin{note}[Podprostor a KPP]
    Je-li $U\subseteq P$ podprostor, pak $(U,\mathcal{L}|_U)$ je konečný projektivní prostor.
\end{note}
\begin{lemma}[Průnik podprostorů je podprostor]
    Pro $U,V\subseteq \mathcal{L}$ je podprostor, pak $U\cap V$ je podprostor.
\end{lemma}
\begin{proof}
    TODO
\end{proof}
\begin{definition}[Obal]
    Buď $A\subseteq P, (P,\mathcal{L})$: pak $\langle A\rangle$ je nejmenší podprostor, který obsahuje $A$.
\end{definition}
\begin{note}[Obal skrze průniky]
    $\langle A\rangle = \bigcap_{U \text{ podprostor} P, A\subseteq U} U$

    $\forall U$ podprostor $P$, pokud $A\subseteq U$, pak $\langle A\rangle\subseteq U$.
\end{note}
\begin{lemma}[O přidání prvku do podprostoru]
    Buď $S\subseteq P$ podprostor $(P,\mathcal{L}), a\not\in S$.
    Potom $\langle S\cup\{a\}\rangle=\bigcup_{x\in S}ax$.
\end{lemma}
\begin{proof}
    TODO
\end{proof}
\begin{lemma}[Sjednocení podprostorů]
    Buďte $S,T\subseteq P$ podprostory $(P,\mathcal{L})$.
    Potom $\langle S\cup T\rangle=\bigcup_{s\in S, t\in T, s\neq t}st$.
\end{lemma}
\begin{proof}
    TODO
\end{proof}
\begin{definition}[Projektivně nezávislá množina]
    Množina $A\subseteq P$ v $(P,\mathcal{L})$ je projektivně nezávislá, jestliže $\forall a\in A: \langle A\setminus \{a\}\rangle\neq\langle A\rangle$.
\end{definition}
\begin{lemma}[Přidání prvku do projektivně nezávislé množiny]
    Je-li $A$ projektivně nezávislá a $b\not\in\langle A\rangle$, pak $A\cup\{b\}$ je projektivně nezávislá.
\end{lemma}
\begin{proof}
    TODO
\end{proof}
\begin{note}[Projektivní nezávislost a generované podprostory]
    $A_1,A_2\subseteq P: \langle A_1\rangle=\langle A_2\rangle\Rightarrow \forall x: A_1\cup\{x\}$ je projektivně nezávislá, právě když $A_2\cup\{x\}$ je projektivně nezávislá.
\end{note}
\begin{proof}
    TODO
\end{proof}
\begin{theorem}[O výměně]
    Buďte $A,B$ projektivně nezávislé množiny v $(P,\mathcal{L})$, $|A|\leq |B|$.
    Pak existuje $b\in B$ taková, že $A\cup\{b\}$ je projektivně nezávislá.
\end{theorem}
\begin{proof}
    TODO
\end{proof}
\begin{definition}[Projektivní báze]
    Projektivní báze je do inkluze maximální projektivně nezávislá množina.
\end{definition}
\begin{consequence}[Projektivně nezávislá množina a báze]
    Každou projektivně nezávislou množinu lze doplnit na bázi a všechny projektivní báze mají stejnou mohutnost.
\end{consequence}
\begin{proof}
    TODO
\end{proof}
\begin{definition}[Dimenze]
    $\dim_P S=|B|-1$, kde $B$ je projektivní báze $S$.
\end{definition}
\begin{note}[O dimenzi]
    \begin{itemize}
        \item $\dim_P(\{a\},\emptyset)=0$
        \item $\dim_P(\emptyset, \emptyset)=-1$
        \item $\dim_P(\text{přímka})=2-1=1$.
        \item $\dim_P(\text{KPR})=2$
    \end{itemize}

    Důsledkem je, že všechny přímky v KPP mají stejnou mohutnost.
\end{note}
\begin{theorem}[Singerova konstrukce]
    Existuje-li $(P,\mathcal{L})$ KPP řádu $q$ dimenze $n$, pak existuje cyklický $(\frac{q^{n+1}-1}{q-1},\frac{q^n-1}{q-1},\frac{q^{n-1}-1}{q-1})$-BIBD.
\end{theorem}
\begin{proof}
    TODO
\end{proof}
\begin{theorem}[O dimenzi průniku a spojení]
    Buďte $U,V$ podprostory $(P,\mathcal{L})$.
    Pak $\dim_P(U\cap V) + \dim_P(U\lor V) = \dim P(U)+\dim_P(V)$.
\end{theorem}
\begin{proof}
    TODO
\end{proof}
\begin{theorem}[Modularita]
    Nechť $A,B,C$ jsou podprostory v $(P,\mathcal{L})$ taková, že $B\subseteq A$.
    Pak $A\cap(B\lor C) = B\lor(A\cap C)$.
\end{theorem}
\begin{proof}
    TODO
\end{proof}
\begin{consequence}[Průnikem dvou rovin v prostoru je přímka]
    Buď $P,\pi,\sigma: \dim_P P=3, \dim_P\pi = \dim_P\sigma=2, \pi\neq\sigma\Rightarrow \pi\cap\sigma$ je přímka -- $\dim_P(\pi\cap\sigma)=1$.
\end{consequence}
\begin{proof}
    TODO
\end{proof}
\begin{theorem}[Roviny si jsou podobné]
    Buď $(P,\mathcal{L})$, s $\pi,\sigma$ rovinami v $P$.
    Pak $\pi\cong\sigma$.
\end{theorem}
\begin{proof}
    TODO
\end{proof}
\begin{theorem}[Nadroviny KPP tvoří BIBD]
    Buď $(P,\mathcal{L})$ konečný projektivní prostor řádu $q$ a dimenze $n$.
    Pak jeho nadroviny tvoří symetrický $(\frac{q^{n+1}-1}{q-1},\frac{q^n-1}{q-1},\frac{q^{n-1}-1}{q-1})$-BIBD.
\end{theorem}
\begin{proof}
    TODO
\end{proof}
\begin{theorem}[KPP dimenze alespoň 3 mají Desargovskou vlastnost]
    Konečné projektivní prostory dimenze alespoň 3 mají Desargovskou vlastnosti.
\end{theorem}
\begin{proof}
    TODO
\end{proof}
\begin{definition}[Automorfismus]
    Bijekce $\alpha: (V,\mathcal{B})\rightarrow (V,\mathcal{B})$ taková, že $\forall B\in \mathcal{B} \alpha[B]\in\mathcal{B}$ je automorfismus.
\end{definition}
\begin{note}[Inverz automorfismu je automorfismus]
    Pro konečné prostory platí, že pro $\alpha$ automorfismus je $\alpha^{-1}$ automorfismus.
\end{note}
\begin{proof}
    TODO
\end{proof}
\begin{definition}[Kolineace]
    Kolineace je zobrazení $\alpha: V\cup \mathcal{B}\rightarrow V\cup \mathcal{B}$ takové, že $\alpha\upharpoonright V$ je bijekce na $V$, $\alpha\upharpoonright \mathcal{B}$ je bijekce na $\mathcal{B}$ a $\forall x\in V, \forall B\in\mathcal{B}: x\in B\Leftrightarrow \alpha(x)\in\alpha(B)$.
\end{definition}
\begin{theorem}[Automorfismy a kolineace]
    Nechť pro $(V,\mathcal{B})$ každé dva různé prvky patří do jednoho bloku a všechny bloky mají mohutnost alespoň 2.
    Nechť $\alpha: V\rightarrow V$ je permutace, $\overline{\alpha}: V\cup B\rightarrow V\cup 2^V$ takové, že $\overline{\alpha}(x)=\alpha(x)$ a $\overline{\alpha}(B)=\alpha[B]$.
    Pak následující jsou ekvivalentní:
    \begin{enumerate}
        \item $\alpha$ je automorfismus
        \item $\overline{\alpha}$ je kolineace
        \item $\alpha$ zachovává kolinearitu
    \end{enumerate}
\end{theorem}
\begin{proof}
    TODO
\end{proof}
\begin{note}[Kolineace a obrazy přímek]
    Je-li $\alpha$ kolineace prostoru, pak $\forall x\neq y\in P: \alpha(xy)=\alpha(x)\alpha(y)$.
\end{note}
\begin{proof}
    TODO
\end{proof}
\begin{definition}[Fixace]
    $A\subseteq P$: $\alpha$ fixuje všechny body $A$, jestliže $\forall x\in A: \alpha(x)=x$

    $l\in\mathcal{L}: \alpha$ fixuje $l$, jestliže $\alpha[l]=l$.
\end{definition}
\begin{lemma}[Kolineace fixující nadrovinu]
    Buď $\alpha$ kolineace fixující všechny body nadroviny $H\subseteq P$.
    Pak existuje $C\in P$ tak, že $\alpha$ fixuje  všechny přímky procházející bodem $C$.
\end{lemma}
\begin{lemma}[Rozšíření kolineace]
    Mějme $q_0\in\mathcal{L}\rightarrow (P',\mathcal{L}'), P' = P\setminus q_0$ množinový systém.
    Pak každou kolineaci $\alpha$ množinového systému $(P',\mathcal{L}')$ je možno rozšířit na kolineaci $\alpha^*$ prostoru $(P, \mathcal{L})$ právě jedním způsobem, a tato kolineace fixuje $q_0$.
\end{lemma}
\begin{definition}[Centrální kolineace]
    Centrální kolineace $(P,\mathcal{L})$ je kolineace, pro niž existuje nadrovina $H$ (nazývaná osa kolineace), jejíž všechny body jsou fixované kolineací $\alpha$, a bod $C\in P$ (nazývaný střed kolineace) takový, že všechny přímky jím procházející jsou zobrazením $\alpha$ fixované.

    (Pozor, může nastat $C\in H$ i $C\not\in H$.)
\end{definition}
\begin{lemma}[Centrální kolineace jsou grupa]
    Centrální kolineace s osou $H$a středem $C$tvoří grupu vzhledem ke skládání, jednotkou je identita.
\end{lemma}
\begin{lemma}[Vlastnosti centrální kolineace]
    Buď $\alpha$ centrální kolineace s osou $H$ a středem $C$. Potom
    \begin{enumerate}
        \item $P\not\in H\cup\{C\}\Rightarrow \forall x: \alpha(x)$ jednoznačně určuje $\alpha_(P): \alpha(x)=CX\cap F\alpha(P)$, kde $F=PX\cap H$
        \item Není-li $\alpha$ identická kolineace, pak každý bod mimo $H\cup \{C\}$ není fixovaný
        \item Centrální kolineace $\alpha$ je jednoznačně určena kteroukoliv dvojicí $P\neq\alpha(P)$.
    \end{enumerate}
\end{lemma}
\begin{consequence}[Jednoznanost středu i osy kolineace]
    Je-li $\alpha$ neidentická kolineace, pak její osa i střed jsou jednoznačně určené.
\end{consequence}
\begin{theorem}[Baerova]
    Buď $H$ nadrovina v Desargovském prostoru $(\mathcal{P},\mathcal{L})$ a buďte $P,P', C$ tři různé kolineární body takové, že $P,P\not\in H$.
    Pak existuje právě jedna centrální kolineace $\alpha$ taková, že $H$ je osa, $C$ je střed a $\alpha(P)=P'$.
\end{theorem}
